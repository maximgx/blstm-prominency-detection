\documentclass[twoside,twocolumn]{article}

\usepackage{blindtext} % Package to generate dummy text throughout this template 

\usepackage[sc]{mathpazo} % Use the Palatino font
\usepackage[T1]{fontenc} % Use 8-bit encoding that has 256 glyphs
\linespread{1.05} % Line spacing - Palatino needs more space between lines
\usepackage{microtype} % Slightly tweak font spacing for aesthetics

\usepackage[italian]{babel} % Language hyphenation and typographical rules
\usepackage[utf8]{inputenc}

\usepackage[hmarginratio=1:1,top=32mm,columnsep=20pt]{geometry} % Document margins
\usepackage[hang, small,labelfont=bf,up,textfont=it,up]{caption} % Custom captions under/above floats in tables or figures
\usepackage{booktabs} % Horizontal rules in tables

\usepackage{lettrine} % The lettrine is the first enlarged letter at the beginning of the text

\usepackage{enumitem} % Customized lists
\setlist[itemize]{noitemsep} % Make itemize lists more compact

\usepackage{abstract} % Allows abstract customization
\renewcommand{\abstractnamefont}{\normalfont\bfseries} % Set the "Abstract" text to bold
\renewcommand{\abstracttextfont}{\normalfont\small\itshape} % Set the abstract itself to small italic text

\usepackage{titlesec} % Allows customization of titles
\renewcommand\thesection{\Roman{section}} % Roman numerals for the sections
\renewcommand\thesubsection{\roman{subsection}} % roman numerals for subsections
\titleformat{\section}[block]{\large\scshape\centering}{\thesection.}{1em}{} % Change the look of the section titles
\titleformat{\subsection}[block]{\large}{\thesubsection.}{1em}{} % Change the look of the section titles

\usepackage{fancyhdr} % Headers and footers
\pagestyle{fancy} % All pages have headers and footers
\fancyhead{} % Blank out the default header
\fancyfoot{} % Blank out the default footer
\fancyhead[C]{Running title $\bullet$ May 2016 $\bullet$ Vol. XXI, No. 1} % Custom header text
\fancyfoot[RO,LE]{\thepage} % Custom footer text

\usepackage{titling} % Customizing the title section

\usepackage{hyperref} % For hyperlinks in the PDF

% Title Section
\setlength{\droptitle}{-4\baselineskip} % Move the title up

\pretitle{\begin{center}\Huge\bfseries} % Article title formatting
\posttitle{\end{center}} % Article title closing formatting
\title{Riconoscimento della Prominenza: Reti Neurali} % Article title
\author{%
\textsc{Maxim Gaina}\thanks{A thank you or further information} \\[1ex] % Your name
\normalsize Università degli Studi di Bologna \\ % Your institution
\normalsize \href{mailto:maxim.gaina@studio.unibo.it}{maxim.gaina@studio.unibo.it}
%\and % Uncomment if 2 authors are required, duplicate these 4 lines if more
%\textsc{Jane Smith}\thanks{Corresponding author} \\[1ex] % Second author's name
%\normalsize University of Utah \\ % Second author's institution
%\normalsize \href{mailto:jane@smith.com}{jane@smith.com} % Second author's email address
}
\date{\today} % Leave empty to omit a date
\renewcommand{\maketitlehookd}{%
\begin{abstract}
\noindent \blindtext % Dummy abstract text - replace \blindtext with your abstract text
\end{abstract}
}

\begin{document}

\maketitle

\section{Introduction}
	\lettrine[nindent=0em,lines=3]{L} orem ipsum dolor sit amet, consectetur adipiscing elit.
	Text requiring further explanation\footnote{Example footnote}.

\section{Lettura}
	\begin{itemize}
		\item definizione formale di prominenza;
		\item \textit{stress} e \textit{intonazione} (\textbf{pitch}, oppure \textbf{tono}) principali attori di collegamento fra prominenza e informazioni fonetico acustiche negli ennunciati;
		\item secondo alcuni studi il \textit{profilo del pitch} (variazione della frequenza fondamentale) risulta essere il più importante indicatore, poi lunghezza e infine intensità;
		\item teoria \textbf{pitch accent} [1958], proposta per stabilire equivalenza netta tra prominenza e fenomeni collegati all'intonazione e quindi alla configurazione assunte dalla frequenza fondamentale; però questo teoria è una presa di posizione troppo netta;
		\item teoria \textbf{metrica} [1975], creazione di albero metrico e griglia metrica, infine creazione della strutture \textit{tune};
		\item \textbf{isocronia} [1979], la prominenza è un fenomeno ritmico che avviene a intervalli di tempo regolari, ma lo studio viene messo in dubbio da alcuni test empirici;
		\item \textit{tone languages} (cinese mandarino) e \textit{stress languages} (inglese), nella prima categoria il pitch determina anche il significato della parola: poi però c'è una via di mezzo che sono le \textit{pitch-accented langauges};
		\item \textit{stress lessica} e \textit{stress frasale};
	\end{itemize}

\section{Results}
	\begin{table}
		\caption{Example table}
		\centering
		\begin{tabular}{llr}
			\toprule
			\multicolumn{2}{c}{Name} \\
			\cmidrule(r){1-2}
			First name & Last Name & Grade \\
			\midrule
			John & Doe & $7.5$ \\
			Richard & Miles & $2$ \\
			\bottomrule
		\end{tabular}
	\end{table}
	\begin{equation}
		\label{eq:emc}
		e = mc^2
	\end{equation}

\section{Concetti Base}
	\subsection{Long Short Term Memory}
		A statement requiring citation \cite{Figueredo:2009dg}.
		\blindtext % Dummy text

	\subsection{Subsection Two}
		\blindtext % Dummy text

\begin{thebibliography}{99}	
	\bibitem{bib:fenomeni-prosodici-prominenza}
		Fabio Tamburini,
		\newblock \emph{Fenomeni Prosodici e Prominenza: Un Approccio Acustico},
		2005.
	
	\bibitem{bib:prominence-detection-italian}
		Fabio Tamburini, Chiara Bertini, Pier Marco Bertinetto,
		\newblock \emph{Prosodic prominence detection in Italian continuous speech using probabilistic graphical models},
		2014
		
	\bibitem{bib:prominence-by-acoustic-analyses}
		Fabio Tamburini,
		\newblock \emph{Automatic Detection of Prosodic Prominence by Means of Acoustic Analyses},
		2015
\end{thebibliography}

\end{document}
